\documentclass[12pt,a4paper]{article}

\usepackage[T1]{fontenc}
\usepackage[polish]{babel}
\usepackage[utf8]{inputenc}
\usepackage{lmodern}
\selectlanguage{polish}
\usepackage{graphicx}
\usepackage{biblatex}
\usepackage{csquotes}
\usepackage{listings}
\usepackage{xcolor}
\usepackage{hyperref}
\usepackage{float}
\usepackage{booktabs}

\addbibresource{bib.bib}

% Konfiguracja listingów kodu
\lstset{
    basicstyle=\ttfamily\small,
    breaklines=true,
    frame=single,
    backgroundcolor=\color{gray!10},
    keywordstyle=\color{blue},
    commentstyle=\color{green!50!black},
    stringstyle=\color{red},
    showstringspaces=false,
    numbers=left,
    numberstyle=\tiny\color{gray}
}

\begin{document}

\nocite{*}

\pagenumbering{gobble}
\clearpage
\begin{figure}[h]
\centering
\includegraphics{media/ps-logo.png}
\end{figure}
\hspace{3cm}
\begin{center}Dokumentacja projektowa\end{center}
\begin{center}CYBER 2025/2026\end{center}
\hspace{3cm}
\begin{center}\large\textbf{System Wykrywania Anomalii w Ruchu Sieciowym}\end{center}
\begin{center}\large\textbf{poprzez analizę portów sieciowych}\end{center}
\hspace{7cm}
\begin{flushright}Kierunek: Informatyka
\end{flushright}
\begin{flushright}Członkowie zespołu:
\par
\textit{Sebastian Pytka}
\par
\textit{Karol Wieczorek}
\par
\textit{Paweł Chwalczyk}
\par
\textit{Adrian Pacyniak}
\end{flushright}
\vfill
\begin{center}Gliwice, 2025/2026\end{center}

\newpage
\pagenumbering{arabic}
\tableofcontents

\newpage
\section{Wprowadzenie}

\subsection{Cel projektu}

Celem projektu jest stworzenie zaawansowanego systemu wykrywania anomalii w ruchu sieciowym z wykorzystaniem analizy portów sieciowych oraz technik uczenia maszynowego. System ma za zadanie identyfikować potencjalne zagrożenia cyberbezpieczeństwa takie jak:

\begin{itemize}
    \item Port scanning (skanowanie portów)
    \item Ataki DDoS (Distributed Denial of Service)
    \item SYN flood
    \item Nietypowe wzorce ruchu sieciowego
    \item Beaconing (komunikacja z serwerami C\&C)
    \item Anomalie statystyczne w wykorzystaniu portów
\end{itemize}

Projekt łączy w sobie metody statystyczne, uczenie maszynowe, reguły heurystyczne oraz analizę temporalną, tworząc wielowarstwowy system detekcji anomalii.

\subsection{Zespół projektowy}

\begin{table}[h]
\centering
\begin{tabular}{lll}
\toprule
\textbf{Imię i nazwisko} & \textbf{Rola} & \textbf{Zadania} \\
\midrule
Sebastian Pytka & Architekt systemu & Koordynacja, architektura \\
Karol Wieczorek & Developer ML & Implementacja ML detektorów \\
Paweł Chwalczyk & Developer Backend & Źródła danych, core logic \\
Adrian Pacyniak & Developer Frontend & Dashboard, wizualizacja \\
\bottomrule
\end{tabular}
\caption{Podział ról w zespole projektowym}
\end{table}

\newpage

\section{Założenia projektowe}

\subsection{Wymagania funkcjonalne}

System powinien umożliwiać:

\begin{enumerate}
    \item \textbf{Analizę ruchu sieciowego} z wielu źródeł:
    \begin{itemize}
        \item Pliki PCAP (analiza offline)
        \item Przechwytywanie pakietów w czasie rzeczywistym (live capture)
        \item Symulator ruchu sieciowego (do testowania)
    \end{itemize}

    \item \textbf{Wykrywanie anomalii} przy użyciu czterech metod:
    \begin{itemize}
        \item Detekcja statystyczna (z-scores, analiza częstotliwości)
        \item Uczenie maszynowe (Isolation Forest, One-Class SVM)
        \item Reguły heurystyczne (port scan, DDoS, SYN flood)
        \item Analiza temporalna (rate limiting, burst detection, beaconing)
    \end{itemize}

    \item \textbf{Wizualizację wyników} poprzez:
    \begin{itemize}
        \item Interaktywny dashboard webowy
        \item Wykresy w czasie rzeczywistym
        \item Live security events feed
    \end{itemize}

    \item \textbf{Generowanie raportów} w formatach:
    \begin{itemize}
        \item JSON (dane strukturalne)
        \item CSV (analiza w arkuszach)
        \item HTML (raporty z wykresami)
    \end{itemize}
\end{enumerate}

\subsection{Wymagania niefunkcjonalne}

\begin{itemize}
    \item \textbf{Wydajność}: Przetwarzanie minimum 1000 pakietów/sekundę
    \item \textbf{Dokładność}: Wskaźnik false positive < 5\%
    \item \textbf{Skalowalność}: Modularna architektura umożliwiająca łatwe dodawanie nowych detektorów
    \item \textbf{Konfigurowalność}: Parametry detekcji konfigurowalne przez YAML
    \item \textbf{Bezpieczeństwo}: Brak wykonywania niebezpiecznych operacji, sandbox dla live capture
\end{itemize}

\subsection{Stos technologiczny}

\begin{table}[h]
\centering
\begin{tabular}{ll}
\toprule
\textbf{Komponent} & \textbf{Technologia} \\
\midrule
Język programowania & Python 3.14 \\
Przechwytywanie pakietów & Scapy \\
Uczenie maszynowe & scikit-learn (Isolation Forest, SVM) \\
Framework webowy & Flask + Socket.IO \\
Wizualizacja & Plotly.js \\
Konfiguracja & PyYAML \\
Logging & Python logging \\
\bottomrule
\end{tabular}
\caption{Wykorzystane technologie}
\end{table}

\newpage

\section{Realizacja projektu}

\subsection{Architektura systemu}

System został zaprojektowany zgodnie z zasadami czystej architektury, z wyraźnym podziałem na warstwy:

\begin{figure}[H]
\centering
\begin{verbatim}
┌─────────────────────────────────────────────────┐
│           Warstwa prezentacji                   │
│  (Dashboard: Flask + Socket.IO + Plotly)        │
└────────────────┬────────────────────────────────┘
                 │
┌────────────────▼────────────────────────────────┐
│         NetworkAnalyzer (Core Logic)            │
│  - Koordynacja detektorów                       │
│  - Zarządzanie danymi                           │
│  - Callback do dashboardu                       │
└────────────────┬────────────────────────────────┘
                 │
        ┌────────┴────────┐
        │                 │
┌───────▼──────┐  ┌───────▼────────────────────────┐
│ Data Sources │  │    Anomaly Detectors           │
│              │  │  1. Statistical Detector       │
│ - PCAP       │  │  2. ML Detector                │
│ - Live       │  │  3. Heuristic Detector         │
│ - Simulator  │  │  4. Temporal Detector          │
└──────────────┘  └────────────────────────────────┘
\end{verbatim}
\caption{Architektura systemu}
\end{figure}

\subsection{Moduły systemu}

\subsubsection{Źródła danych (Data Sources)}

\paragraph{PCAPReader}
Moduł do odczytu i analizy plików PCAP. Wykorzystuje bibliotekę Scapy do parsowania pakietów sieciowych i konwersji ich do wewnętrznego formatu \texttt{NetworkPacket}.

\begin{lstlisting}[language=Python, caption=Klasa NetworkPacket]
@dataclass
class NetworkPacket:
    timestamp: datetime
    src_ip: str
    dst_ip: str
    src_port: int
    dst_port: int
    protocol: str  # TCP, UDP, ICMP
    packet_size: int
    flags: str
\end{lstlisting}

\paragraph{LiveCapture}
Moduł do przechwytywania pakietów w czasie rzeczywistym. Wymaga uprawnień administratora (sudo) do dostępu do interfejsu sieciowego.

\paragraph{TrafficSimulator}
Generator syntetycznego ruchu sieciowego do testowania. Generuje zarówno normalny ruch, jak i symulowane anomalie (port scans, burst traffic).

\subsubsection{Detektory anomalii}

\paragraph{Statistical Detector}
Wykorzystuje metody statystyczne do wykrywania anomalii:

\begin{itemize}
    \item \textbf{Z-score}: Oblicza odchylenie standardowe dla częstotliwości portów, rozmiarów pakietów i adresów IP
    \item \textbf{Sliding window}: Analizuje ostatnie N pakietów (domyślnie 100)
    \item \textbf{Progi}: Konfigurowalny threshold (domyślnie 5.0)
\end{itemize}

Wzór na z-score:
\[ z = \frac{x - \mu}{\sigma} \]

gdzie $x$ to obserwowana wartość, $\mu$ to średnia, a $\sigma$ to odchylenie standardowe.

\paragraph{ML Detector}
Wykorzystuje dwa algorytmy uczenia maszynowego:

\textbf{1. Isolation Forest} \cite{Liu2008IsolationForest}
\begin{itemize}
    \item Algorytm wykrywania outlierów oparty na drzewach decyzyjnych
    \item Kontaminacja: 3\% (oczekiwany procent anomalii)
    \item Liczba estymatorów: 100
\end{itemize}

\textbf{2. One-Class SVM}
\begin{itemize}
    \item Algorytm klasyfikacji binarnej (normalny/anomalny)
    \item Parametr nu: 0.03
    \item Kernel: RBF (Radial Basis Function)
    \item Implementacja z scikit-learn \cite{ScikitLearn}
\end{itemize}

\textbf{Ekstrakcja cech}:
\begin{lstlisting}[language=Python, caption=Ekstrakcja cech dla ML]
features = [
    src_port,      # Port źródłowy
    dst_port,      # Port docelowy
    packet_size,   # Rozmiar pakietu
    protocol_num,  # Protokół (TCP=6, UDP=17, ICMP=1)
    hour           # Godzina (0-23)
]
\end{lstlisting}

\textbf{Whitelist system}:
Aby zredukować false positive, zaimplementowano system whitelist:
\begin{itemize}
    \item Znane zakresy IP (Google, Apple, Microsoft, AWS, GitHub)
    \item Standardowe porty (443/HTTPS, 80/HTTP, 53/DNS, 5353/mDNS, 1900/SSDP)
    \item Adresy multicast (224.*, 239.*)
    \item Minimum confidence threshold: 0.3
\end{itemize}

\paragraph{Heuristic Detector}
Implementuje reguły wykrywania znanych wzorców ataku:

\textbf{1. Port Scan Detection}:
\begin{itemize}
    \item Próg: 10 unikalnych portów / 5 sekund z jednego IP
    \item Wykrywa techniki: SYN scan, TCP connect scan
\end{itemize}

\textbf{2. DDoS Detection}:
\begin{itemize}
    \item Próg: 100 połączeń / sekundę do jednego IP
    \item Wykrywa ataki volumetryczne
\end{itemize}

\textbf{3. SYN Flood Detection}:
\begin{itemize}
    \item Analiza ratio pakietów SYN do SYN-ACK
    \item Próg: > 80\% pakietów to SYN
\end{itemize}

\textbf{4. Unusual Ports}:
\begin{itemize}
    \item Wykrywa połączenia do nietypowych portów (> 49152)
    \item Może wskazywać na backdoory lub malware C\&C
\end{itemize}

\paragraph{Temporal Detector}
Analizuje wzorce czasowe w ruchu sieciowym:

\textbf{1. Rate Limiting Violations}:
\begin{itemize}
    \item Wykrywa przekroczenie limitu pakietów/sekundę
    \item Domyślny limit: 50 pkt/s
\end{itemize}

\textbf{2. Burst Detection}:
\begin{itemize}
    \item Wykrywa nagłe skoki w ruchu
    \item Próg: 200 pakietów w krótkim czasie
\end{itemize}

\textbf{3. Beaconing Detection}:
\begin{itemize}
    \item Wykrywa regularne, okresowe połączenia (np. malware C\&C)
    \item Analiza interwałów czasowych między pakietami
\end{itemize}

\textbf{4. Off-hours Activity}:
\begin{itemize}
    \item Wykrywa nietypową aktywność w godzinach 2:00-5:00
    \item Może wskazywać na nieautoryzowane działania
\end{itemize}

\subsection{Dashboard i wizualizacja}

Dashboard webowy został zaimplementowany w Flask z wykorzystaniem Socket.IO do komunikacji w czasie rzeczywistym.

\subsubsection{Funkcjonalności dashboardu}

\begin{itemize}
    \item \textbf{Real-time statistics}: Całkowita liczba pakietów, anomalii, detection rate
    \item \textbf{Wykresy interaktywne}:
    \begin{itemize}
        \item Timeline anomalii (scatter plot)
        \item Rozkład typów anomalii (pie chart)
        \item Rozkład severity (bar chart)
        \item Top anomaly sources (bar chart)
    \end{itemize}
    \item \textbf{Live security events}: Feed z ostatnimi wykrytymi anomaliami
    \item \textbf{Auto-refresh}: Automatyczna aktualizacja co 2 sekundy
\end{itemize}

\subsection{System raportowania}

Moduł \texttt{ReportGenerator} generuje trzy typy raportów:

\paragraph{JSON Report}
\begin{itemize}
    \item Struktura danych z pełnymi informacjami o anomaliach
    \item Wykorzystywany do dalszej analizy programowej
    \item Zawiera: statystyki, lista anomalii, metadata
\end{itemize}

\paragraph{CSV Report}
\begin{itemize}
    \item Format tabelaryczny do analizy w Excel/Pandas
    \item Kolumny: timestamp, type, severity, source\_ip, destination\_ip, port, confidence
\end{itemize}

\paragraph{HTML Report}
\begin{itemize}
    \item Raport z interaktywnymi wykresami Plotly
    \item Zawiera wizualizacje i podsumowanie
    \item Gotowy do prezentacji
\end{itemize}

\subsection{Konfiguracja systemu}

Wszystkie parametry systemu są konfigurowalne przez plik YAML:

\begin{lstlisting}[language=yaml, caption=Przykladowa konfiguracja (config.yaml)]
# Zrodlo danych
data_source:
  mode: "simulator"  # pcap, live, simulator
  pcap_file: "data/sample_traffic.pcap"
  network_interface: "en0"

# Detekcja
detection:
  statistical:
    enabled: true
    z_score_threshold: 5.0
    window_size: 100

  ml:
    enabled: true
    isolation_forest:
      contamination: 0.03
      n_estimators: 100
    one_class_svm:
      nu: 0.03
      gamma: 'auto'

  heuristic:
    enabled: true
    port_scan:
      threshold: 10
      time_window: 5

  temporal:
    enabled: true
    rate_limit: 50
    burst_threshold: 200

# Dashboard
dashboard:
  host: "127.0.0.1"
  port: 5000
  debug: true
  auto_refresh: 2
\end{lstlisting}

\newpage

\section{Testy i wyniki}

\subsection{Metodologia testowania}

System został przetestowany w trzech scenariuszach:

\begin{enumerate}
    \item \textbf{Symulator ruchu}: 1000 pakietów z 10\% wygenerowanych anomalii
    \item \textbf{Analiza PCAP}: Pliki z rzeczywistego ruchu sieciowego
    \item \textbf{Live capture}: Przechwytywanie w czasie rzeczywistym
\end{enumerate}

\subsection{Optymalizacja false positive rate}

Jednym z głównych wyzwań było zredukowanie liczby fałszywych alarmów. Przed optymalizacją system wykrywał 42\% pakietów jako anomalie, z czego większość stanowił legalny ruch (Google, Apple, Microsoft).

\paragraph{Zmiany optymalizacyjne}:

\begin{table}[H]
\centering
\begin{tabular}{lll}
\toprule
\textbf{Parametr} & \textbf{Przed} & \textbf{Po} \\
\midrule
Contamination (IF) & 0.01 & 0.03 \\
Nu (SVM) & 0.01 & 0.03 \\
Whitelist IP & Brak & 10+ zakresów \\
Whitelist portów & 3 & 11 \\
Min confidence & Brak & 0.3 \\
\bottomrule
\end{tabular}
\caption{Zmiany w parametrach ML detektora}
\end{table}

\paragraph{Wyniki optymalizacji}:

\begin{table}[H]
\centering
\begin{tabular}{lrr}
\toprule
\textbf{Metryka} & \textbf{Przed} & \textbf{Po} \\
\midrule
Detection rate & 42.0\% & 3.2\% \\
ML SVM anomalie & 689 & 0 \\
ML Isolation anomalie & 248 & 50 \\
False positive & $\sim$95\% & $\sim$5\% \\
\bottomrule
\end{tabular}
\caption{Porównanie wyników przed i po optymalizacji}
\end{table}

\textbf{Redukcja false positive: 92\%}

\subsection{Analiza wydajności}

\begin{table}[H]
\centering
\begin{tabular}{lr}
\toprule
\textbf{Metryka} & \textbf{Wartość} \\
\midrule
Throughput & $\sim$1500 pakietów/sekundę \\
Latencja detekcji & < 10 ms \\
Wykorzystanie CPU & $\sim$15-25\% \\
Wykorzystanie RAM & $\sim$150-200 MB \\
\bottomrule
\end{tabular}
\caption{Charakterystyka wydajnościowa systemu}
\end{table}

\subsection{Przykłady wykrytych anomalii}

\paragraph{Port Scan}:
\begin{verbatim}
Type: port_scan
Source: 192.168.1.100
Unique ports: 25 w 3 sekundy
Severity: HIGH
Confidence: 0.95
\end{verbatim}

\paragraph{Nietypowy port}:
\begin{verbatim}
Type: unusual_port
Source: 192.168.1.37
Destination: 208.103.161.2:65283
Severity: MEDIUM
Confidence: 0.76
\end{verbatim}

\paragraph{ML Isolation}:
\begin{verbatim}
Type: ml_isolation
Packet size: 1242 bytes
Protocol: UDP
Anomaly score: -0.688
Confidence: 0.34
\end{verbatim}

\newpage

\section{Podsumowanie i wnioski}

\subsection{Podsumowanie}

Projekt zakończył się sukcesem - udało się zrealizować w pełni funkcjonalny system wykrywania anomalii w ruchu sieciowym, który:

\begin{itemize}
    \item \textbf{Integruje cztery metody detekcji}: statystyczną, ML, heurystyczną i temporalną
    \item \textbf{Przetwarza dane z trzech źródeł}: PCAP, live capture, simulator
    \item \textbf{Osiąga niski wskaźnik false positive}: 3.2\% (po optymalizacji)
    \item \textbf{Dostarcza intuicyjny interface}: real-time dashboard z wykresami
    \item \textbf{Generuje profesjonalne raporty}: JSON, CSV, HTML
    \item \textbf{Jest konfigurowalny}: parametry w YAML, modularna architektura
\end{itemize}

System został przetestowany na rzeczywistym ruchu sieciowym i skutecznie wykrywa:
\begin{itemize}
    \item Port scanning
    \item Ataki DDoS
    \item Nietypowe wzorce komunikacji
    \item Potencjalny malware (beaconing, unusual ports)
\end{itemize}

\subsection{Wnioski}

\subsubsection{Mocne strony projektu}

\begin{itemize}
    \item \textbf{Wielowarstwowa detekcja}: Kombinacja różnych metod zwiększa skuteczność wykrywania
    \item \textbf{Whitelist system}: Drastycznie redukuje false positive dla znanego ruchu
    \item \textbf{Modularna architektura}: Łatwe dodawanie nowych detektorów
    \item \textbf{Real-time processing}: Możliwość analizy na żywo z dashboardem
    \item \textbf{Machine Learning}: Wykrywa nieznane wcześniej wzorce anomalii
\end{itemize}

\subsubsection{Wyzwania i rozwiązania}

\begin{enumerate}
    \item \textbf{Problem}: Wysoki false positive rate (42\%)
    \par \textbf{Rozwiązanie}: Implementacja whitelist, zwiększenie parametrów ML, minimum confidence threshold

    \item \textbf{Problem}: Trudność w odróżnieniu normalnego HTTPS od anomalii
    \par \textbf{Rozwiązanie}: Whitelist znanych dostawców (Google, Apple, etc.), port 443 na whitelist

    \item \textbf{Problem}: Lokalne protokoły (mDNS, SSDP) wykrywane jako anomalie
    \par \textbf{Rozwiązanie}: Dodanie portów 5353, 1900 i adresów multicast do whitelist

    \item \textbf{Problem}: Live capture wymaga uprawnień root
    \par \textbf{Rozwiązanie}: Dokumentacja z instrukcjami sudo, alternatywne tryby (PCAP, simulator)
\end{enumerate}

\subsubsection{Możliwe rozszerzenia}

Kierunki dalszego rozwoju systemu:

\begin{itemize}
    \item \textbf{Deep Packet Inspection}: Analiza payload pakietów (obecnie tylko header)
    \item \textbf{Distributed deployment}: Agregacja danych z wielu sensorów
    \item \textbf{Alert system}: Powiadomienia email/SMS/Slack o krytycznych anomaliach
    \item \textbf{Database backend}: PostgreSQL/InfluxDB do długoterminowego przechowywania
    \item \textbf{Advanced ML}: Deep Learning (LSTM, Autoencoders) dla sekwencji pakietów
    \item \textbf{Integration}: API do integracji z SIEM (Splunk, ELK Stack)
    \item \textbf{Threat Intelligence}: Integracja z bazami IoC (Indicators of Compromise)
    \item \textbf{Automated response}: Automatyczne blokowanie podejrzanych IP
\end{itemize}

\subsubsection{Wnioski końcowe}

Projekt pokazuje, że skuteczna detekcja anomalii w ruchu sieciowym wymaga:

\begin{enumerate}
    \item \textbf{Hybrydowego podejścia}: Żadna pojedyncza metoda nie jest wystarczająca
    \item \textbf{Balansu}: Między wrażliwością detekcji a liczbą false positive
    \item \textbf{Kontekstu}: Whitelist i domain knowledge znacząco poprawiają dokładność
    \item \textbf{Iteracyjnej optymalizacji}: Parametry ML wymagają dostrojenia do konkretnej sieci
\end{enumerate}

System stanowi solidną bazę do dalszych badań i rozwoju w obszarze cyberbezpieczeństwa, demonstrując praktyczne zastosowanie uczenia maszynowego w wykrywaniu zagrożeń sieciowych.

\newpage
\section{Bibliografia}

\printbibliography[heading=none]

\newpage
\appendix
\section{Instrukcja uruchomienia}

\subsection{Wymagania systemowe}

\begin{itemize}
    \item Python 3.14 lub nowszy
    \item Uprawnienia administratora (dla live capture)
    \item System operacyjny: Linux, macOS lub Windows
\end{itemize}

\subsection{Instalacja}

\begin{lstlisting}[language=bash, caption=Kroki instalacji]
# 1. Klonowanie repozytorium
git clone <repository-url>
cd netport-anomaly-detector

# 2. Utworzenie srodowiska wirtualnego
python3 -m venv .venv
source .venv/bin/activate  # Linux/macOS
# lub
.venv\Scripts\activate     # Windows

# 3. Instalacja zaleznosci
pip install -r requirements.txt
\end{lstlisting}

\subsection{Uruchomienie}

\begin{lstlisting}[language=bash, caption=Przykłady uruchomienia]
# Tryb simulator (testowanie)
python main.py

# Analiza pliku PCAP
python main.py --mode pcap --pcap-file data/capture.pcap

# Live capture (wymaga sudo)
sudo python main.py --mode live --interface en0

# Bez dashboardu (tylko CLI)
python main.py --no-dashboard

# Generowanie raportow
python main.py --report-only
\end{lstlisting}

\subsection{Dostęp do dashboardu}

Po uruchomieniu, dashboard jest dostępny pod adresem:
\begin{center}
\url{http://127.0.0.1:5000}
\end{center}

\end{document}
